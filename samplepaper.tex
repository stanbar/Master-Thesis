\documentclass[runningheads]{llncs}
\usepackage{graphicx}
\begin{document}
\title{Application of blockchain and infection processes in graphs to
mitigate content poisoning attacks in information-centric networks}
\author{Stanisław Barański, Jerzy Konorski}
\institute{Faculty of Electronics, Telecommunications and Informatics\\
Gdansk University of Technology\\
Gdansk, Poland\\
\email{s160518@student.pg.edu.gda.pl}}
\maketitle
\begin{abstract}
Information-centric networks introduce new vector of attacks, one of them is content poisoning, which when performed successfully, can create destructive damages. Currently known authentication methods like login/password, private key, biometry, SMS/email confirmation operate on the same dimension of authentication. We propose another dimension of authentication which is time availability. When intruder publisher is operating in time-constrained environment, his access to target identity is limited, where such constrains doesn't apply to honest publisher. And such distinguish we leverage to propose new authentication mechanism. Two implementations are proposed, first one is based on infection processes in graphs and second one is backed by blockchain technology. 

\keywords{Content Poisoning \and Blockchain \and ICN  \and Graph infection}
\end{abstract}

\section{Introduction - State of the art}
Current internet architecture is mostly based on TCP/IP stack, which allow  establish communication channels between two IP addresses. While it worked great for past years,  it struggle to fit current demands. Today internet is dominated by transporting content such as audio, video, images and text[source] from content creators to content consumers.
% Write more about how the internet is used today
% Write more about why current architecture doesn't fit 
% ICN - the solution !!!
The Information-centric networks like NDN, DONA or IPFS \cite{benet2014ipfs} introduce paradigm shift from host-centric to content-oriented paradigm. Placing content in the center of interest allow us to achieve number of benefits. All network participates become more aware of the transferring content. This kind of awareness allows to implement various improvements over host-based paradigm such as content caching, mobility, integrity and security assurance. All of those features are guaranteed naively by ICN transport layer in contrast to TCP/IP where we needed to build them on top of it.
% continue with Content protection in NDN

\section{Content Poisoning - Problem}
The most significant benefit of ICN networks is content caching. It turns out that this benefit becomes double-edged sword when we take into account security concerns. Attacker who successfully performs CPA (content-poisoning attack) on one CS (content cache), may achieve destructive success 
% can we say that?.
By attacker we understand malicious provider who gets access to a "signing machine" used by legitimate provider. In consequences it's impossible just by verifying packet signature to determine if the content is poisoned or not. NDN network is designed to ease content diffusion, not matter if genuine or not.

\section{Wikipedia}
Wikipedia is great example of system that face with content poisoning problem. Let's investigate how do they solve the problem. 
Everyone can become Wikipedia editor just by creating free account. Account reqistration doesn't even require email address. Editor can create new article or modify existing one. But before the change is publicly available, other editors read the proposition, and have a chance to reject it or propose further edit. If they can not achieve consensus, they start a discussion where they exchange their arguments, finally if they agree on one version, consensus is achieved and new version of article is published (keeping the whole history of changes).
% What are the kinds of sockpuppeting 
% How to fight with sockpuppeting
% https://en.wikipedia.org/wiki/Wikipedia:Sock_puppetry
% they take into account the 5 phillars  etc. etc.
Such ease of account creation makes Wikipedia vulnerable to   Sockpuppeting (also known as Sybil attack). This breach allows one user to create multiple accounts with different identity. In consequence, one person control fake "public opinion", which can support his position in edits discussions. That's why raw voting system is non-prefered method in conflict resolutions. Wikipedia consensus is rather collaborative, in contrast to competition consensus used in Bitcoin, where only one account who first find the proof-of-work wins all bitcoin reward. 
Wikipedia fight with content poisoning attacks, by using human to detect bogus changes. This mechanism seems to work well in such service, but it doesn't scale to public internet protocol.

\section{Bitcoin}
It turns out that bitcoin is one of the most successfully mantained decentralized system. Thus it's worth investigation on how they fight with CPA problem. We will look at two aspects of Bitcoin, first is project development, and second is Blockchain consensus.

\subsection{Project Development}

\subsection{Blockchain consensus}

Let's look how other content centric networks solve the problem of content posioning. We will start with Wikipedia. Each article creation or modification needs to be agreed by consensus.  
When we think about our problem it seems to be quite similar to content posioning on Wikipedia. 

\section{Blockchain}
When we speak about trust, openess and decentralization it's always worth to consider Blockchain technology. Here we will try to use Blockchain to achieve previously stated requirements. Some of the properties Blockchain technology gives us are: immutability - once written, data can not be modified. Time assurance - each block is published in apprx. 10min. This feature gives us a clock that will provide us signed timestamps. Sign ensure trust based on consensus settled by proof-of-work algorithm.

\section{Proof of Time - Abstract solution}
We propose system where authentication is based on \textit{proof of time} access to private keys. 
We noticed that the proof-of-time must be published in either neutral place or in verifier place. So the publisher can not use it's influence to modify the data. 

\section{Trust graph}
How are they created ? In life, in networks ? What does it mean to create trust relationship. What do we understand by trust relationship. In environments that or another. 

Then we assume that such relation is already achieved. 
LOCKSS - solve the problem in one way, Wikipedia in another (experts consensus), but they doesn't fit to us, why ?. 

Why do we even look for solution for our problem in the context of human trust ? It turns out that people are still one of the most advanced technology, thus we can get inspired by some of the solutions that worked in human societies for centuries. Let's dive into it.
Yuval Noah Harari in his book "Sapiens: A Brief History of Humankind" states that the most important feature of human language is rumor. Rumor let us know which person is not trustworthy without having to interact with him directly. If our best friend Bob, tells us, that Carlie is theft, we don't need to get stolen to be convinced about it. Same applies in inverse scenario, if Bob tells us, that Carlie, sells great quality products, we are now more likely to buy some products from him. We notice that each person we know directly or indirectly, gets labeled by some tags. One can be labeled as Helpful, Conscientiousness and also Not-Trustworthy, while other can be labeled as Unhelpful, Lazy but Trustworthy. Here is this paper we are limiting our range of study just to dimension of trust.
What if we have three friends Alice, Bob, Charlie. Alice and  Bob tells us that David is trustworthy, while Charlie claims that he's not. Decision of labeling David as trusted or not, requires some kind of decision evaluation algorithm.
One might assume that if there is at least one person who doesn't trust him, there must be something wrong with him, and will label him as Not-Trustworthy. One can use the most natural to human beings evaluator which says, do want majority of people do, thus if Charlie is trusted by majority, I will trust her too. Another one can slightly generalise this evaluator and say that person is trustworthy, only if $\xi$ percentage of my friends trust him. 

At this point it's worth introducing some convention. When we say friend we mean a trustworthy person, in other words, person to who we have trust relation to. Let $N$ to be a set of all considered person group, friends and non friends. Let $F$ be a set of all our friends $f \in F$. Let $F_n$ be a subset of $F$ where all $f$ trust person $n$. Then we will call $\%_n = |F_n|/|F|$ the proportion of our fiends who trust particular person $n$. Let's call $\xi$ (where $0 \le \xi \leq 1$) the minimum proportion of our friends $\%_n$ who needs to trust person $n$ in order to make me trust him. 

So as we said previously that we will trust David only if majority of our friends trust him. We denote trust function as $T(n) = \%_n > \xi : (N) \rightarrow \{0,1\}$. 
Let's use this formula to evaluate if $David$ is trustworthy person. Let $\xi = 0.5$. We know that Alice and Bob do trust David, while Charlie don't.
\[T(David) = \%_{David} > \xi\]
\[= |F_n|/|F| > \xi\]
\[= \frac{2}{3} > \frac{1}{2}\]
\[= 1\]
Then it turns out that $David$ is \textbf{Trustworthy}


People with low $\xi$ easly gets manipulated, we call them naive.
People with high $\xi$ hardly gets convinced, we call them stubborn. 

Another generalisation might be adding weights to this evaluator, let's say that Charlie is our brother, while Alice and Bob are our cousins, and we trust 3 times stronger to our brother than cousin. Let's call $W(f): (F) \rightarrow \Re$ the function that maps our friend to the weight of how strong we trust him. In this case weighted proportion of our friends \[\%_n = \frac{\sum\limits_f^{F_n} W(f)}{\sum\limits_f^{F} W(f)}\]

When we assume weights $W(Alice) = 1, W(Bob) = 1, W(Charlie) = 3$, and $\xi = 0.5$. We can calculate weighted proportion $T(David)$ as follows:
\[T(David) = \%_n > \xi\]
\[= \frac{\sum\limits_f^{F_n} W(f)}{\sum\limits_f^{F} W(f)} > \xi\]
\[= \frac{\sum \{1,1\}}{\sum\{1,1,3\}} > \frac{1}{2}\]
\[= \frac{2}{5} > \frac{1}{2}\]
\[= 0\]
Then it turns out that $David$ is \textbf{Not-Trustworthy}

But this view is based on static network of connections. We somehow meet Alice, Bob and Charlie, and we gets convinced about their trustworthness. Thus there must be a second way of gaining our trust. Let's modify our trust function by allowing External Trust Obtaining(ETO). $T(n) = \%_n > \xi \lor ETO(n) : (N) \rightarrow \{0,1\}$. 

Another think we can observe in the context of trust network is time. Should we still trust our friend from elementary school if we haven't seen him for decades ? We argue that trust is function of time.





\section{Stolen Credentials Problem}
How long on average does it take to detect stolen credentials on average.

\section{Graph Burning}
How does it differ from out solution ? Where to place fire so it burn as fast as possible ?


\section{Graph Infections - Concrete solution}
Proposition \cite{jekon2019content}

The graph we are considering is limited only to one content at a time. So we can assume that all nodes in the graph are interested in such content trust.

\section{Blockchain - Concrete solution}
Blockchain requirements.
\begin{itemize}
\item \textbf{Open and permissionless}
We argue that if such blockchain would have to operate in open internet, it also have to be open, so such blochain have to be completely permissionless, and allow to join new participants freely. 
\item \textbf{Free}
Most of the open permissionless blockchain charge fee for transaction. It incentivise participants to mantain network nodes.
Here we have different kind of incentiviation, that is trust. 
\item \textbf{Lightweight} This software is will be running of software devices, which generally resources constrained. But we must remember that NDN nodes are also more resource hungry than typical tcp/ip routers.
\item \textbf{Scalable} If we think of implementing such software on global scale, it must be scalable. 
\end{itemize}


\section{Comparison}
\begin{table}[]
\begin{tabular}{lll}
                                   & Graph Infections & Blockchain \\
Byzantine tolerant system          & False            & True       \\
Determined                         & False            & True       \\
Memory expensive                   & True             & True       \\
Progressive trust                  & False            & True       \\
Transport layer dependent          & True             & Optional   \\
Require Network topology awareness & True             & False     
\end{tabular}
\end{table}


\section{Discussion}

%
% ---- Bibliography ----
%
% BibTeX users should specify bibliography style 'splncs04'.
% References will then be sorted and formatted in the correct style.
%
\bibliographystyle{splncs04}
\bibliography{refs}

%
\end{document}

