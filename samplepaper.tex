\documentclass[runningheads]{llncs}
\usepackage{graphicx}
\begin{document}
\title{Application of blockchain and infection processes in graphs to
mitigate content poisoning attacks in information-centric networks}
\author{Stanisław Barański, Jerzy Konorski}
\institute{Faculty of Electronics, Telecommunications and Informatics\\
Gdansk University of Technology\\
Gdansk, Poland\\
\email{s160518@student.pg.edu.gda.pl}}
\maketitle
\begin{abstract}
Information-centric networks introduce new vector of attacks, one of them is content poisoning, which when performed successfully, can create destructive damages. Currently known authentication methods like login/password, private key, biometry, SMS/email confirmation operate on the same dimension of authentication. We propose another dimension of authentication which is time availability. When intruder publisher is operating in time-constrained environment, his access to target identity is limited, where such constrains doesn't apply to honest publisher. And such distinguish we leverage to propose new authentication mechanism. Two implementations are proposed, first one is based on infection processes in graphs and second one is backed by blockchain technology. 

\keywords{Content Poisoning \and Blockchain \and ICN  \and Graph infection}
\end{abstract}

\section{Introduction - State of the art}
Current internet architecture is mostly based on TCP/IP stack, which allow  establish communication channels between two IP addresses. While it worked great for past years,  it struggle to fit current demands. Today internet is dominated by transporting content such as audio, video, images and text[source] from content creators to content consumers.
% Write more about how the internet is used today
% Write more about why current architecture doesn't fit 
% ICN - the solution !!!
The Information-centric networks like NDN, DONA or IPFS \cite{benet2014ipfs} introduce paradigm shift from location addressing to content addressing. Placing content in the center of interest allow us to achieve number of benefits. All network participates become more aware of the transferring content. This kind of awareness allows to implement various improvements over host-based paradigm such as content caching, mobility, integrity and security assurance. All of those features are guaranteed naively by ICN transport layer in contrast to TCP/IP where we needed to build them on top of it.
% continue with Content protection in NDN

\section{Content Poisoning - Problem}
The most significant benefit of ICN networks is content caching. It turns out that this benefit becomes double-edged sword when we take into account security concerns. Attacker who successfully performs CPA (content-poisoning attack) on one CS (content cache), may achieve destructive success 
% can we say that?.
By attacker we understand malicious provider who gets access to a "signing machine" used by legitimate provider. In consequences it's impossible just by verifying packet signature to determine if the content is poisoned or not. NDN network is designed to ease content diffusion, not matter if genuine or not.

\section{Proof of Time - Abstract solution}
We propose system where authentication is based on \textit{proof of time} access to private keys. 

\section{Graph Infections - Concrete solution}
Proposition \cite{jekon2019content}
\section{Blockchain - Concrete solution}

\section{Discussion}

%
% ---- Bibliography ----
%
% BibTeX users should specify bibliography style 'splncs04'.
% References will then be sorted and formatted in the correct style.
%
\bibliographystyle{splncs04}
\bibliography{refs}

%
\end{document}

