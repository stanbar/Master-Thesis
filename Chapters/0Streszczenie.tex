\chapter*{Streszczenie}
Sieci informacjo-centryczne (ang. Information-centric Networking (ICN))––które kandydują do miana sieci przyszłości––zmagają się z nowym wektorem ataków. Jednym z ataków jest atak zatruwania treści (ang. Content Posioning Attack (CPA)), a zwłaszcza jego silniejsza forma––atak zatruwania treści przez fałszywe dane (ang. Fake Data CPA). Sieci te, są wyjątkowo podatne na ten typ ataków, ponieważ nie pozwalają na usuwanie treści––w momencie, gdy treść trafia do sieci, nie jest możliwe jej usunięcie inaczej niż poprzez wygaśnięcie.
Aktualnie znane metody uwierzytelniania jak login/hasło, klucze prywatne, biometria, potwierdzenie SMS/email operują w jednym tym samym wymiarze uwierzytelniania. W tej pracy proponujemy nowy wymiar uwierzytelniania, który bazuje na dostępnie do czasu. Kiedy napastnik-wydawca operuje w czasowo ograniczonym środowisku, jego dostęp do skradzionych uwierzytelnień jest czasowo ograniczony, podczas gdy prawdziwy wydawca nie jest ograniczony czasowo w żaden sposób. Ta różnica jest wykorzystana w celu stworzenia nowego systemu uwierzytelniania.
Dwie implementacje systemu są zaproponowane, pierwsza bazuje na algorytmie infekcji na grafach, a druga na łańcuchu bloków (ang. Blockchain). Dzięki zbudowanym symulatorom jesteśmy w stanie zaobserwować problem Sojuszu Obronnego (ang. Deffensive Alliance) w algorytmie infekcji na grafach.
Obie metody zostają ocenione pod kątem złożoności komunikacyjnej, determinizmowi algorytmów, możliwości dopasowania się do zmian, oraz odporności na błędy (również silniejszej formy bizantyjskich generałów).