
\chapter{Comparison}
Let's try to compare those two mechanisms. 

First of all, the GI in its raw form doesn't handle faulty nodes––both in stop and Byzantine fault form. Most of the blockchains tolerate up to $\frac{n-1}{3}$ Byzantine-faulty nodes, and Stellar FBA is not different here. A blockchain solution is better in this aspect.

Along with the Byzantine-fault tolerance, comes the PKI overhead. Therefore the Blockchains solution require PKI infrastructure

GI allows to prune its trust table as the content is being removed from Content Store, so there is no need to store the trust information to content that is no longer stored. Blockchain on the other hand is permanent, and in order to be consistent no data can be removed (otherwise the chain of hashes would be broken). 


GI method also require topology awareness. All nodes in the network should know all other nodes in the network––or at least some of them. It needs to know all nodes in the network to select random node where publisher can infect it for next infection


Another property is determinism, the GA algorithm is highly random when it comes to a consensus decision. Even a high number of proof-of-times can sometimes result in graph extinction which we consider unwanted property. Blockchain solution on the other side is deterministic, a publisher with a high number of proof-of-time claims can be completely sure that his content gets authenticated. 

Resilience is another aspect worth comparing. GI algorithm is sensitive to both randomness, network structure, network size, $\xi$, and $Z_{IA}$ parameters. While the blockchain––particularly FBA––has just one assumption about the quorums-intersection. We state that the blockchain solution is more resilient. 

Another aspect is communication complexity, in GI each node has to communicate just with its neighbors so the complexity is constant $O(1)$, while in blockchain solutions the communication has to be made with all nodes in the network $O(n)$. In this matter, the GI algorithm wins.

GI allows to prune its trust table as the content is being removed from Content Store, so there is no need to store the trust information to content that is no longer maintained. Blockchain on the other hand is completely immutable, and the data once stored can not be removed. Blockchain storage is permament. in order to be consistent no data can be removed (otherwise the chain of hashes would be broken). 




We present comparison in the Table \ref{table:comparison} present the comparison in tabular format.

\begin{table}[h!]
\centering
\begin{tabular}{ || m{3cm} | m{6cm}| m{6cm} || } 
\hline
\hline
Property & Graph Infection & Blockchain \\ 
\hline\hline
Fault-tolerance & None & Byzantine-Fault tolerance  \\ 
\hline
Outcome & Probabilistic & Deterministic  \\ 
\hline
Resilience & Depend on randomness, network structure, network size, $\xi$, and $Z_{IA}$ parameters & Depend on network structure (quorums-intersection)  \\ 
\hline
Communication complexity & O(1) & O(n)  \\ 
\hline
PKI required & No & Yes  \\ 
\hline
Storage & Prunable & Permanent  \\ 
\hline
Throughput & No & Yes \\
\hline
\end{tabular}
\caption{Comparison of GI and Blockchain algorithms}
\label{table:comparison}
\end{table}

\chapter{Summary}
We started this paper from formulating the problem of Fake Data CPA. It turns out that Fake Data CPA is a hard problem, but assuming certain conditions (time constraints), we proved that it is solvable. Time constrain allows us to extend the authentication from \textit{access to credentials} to both \textit{access to credentials} and \textit{access to time}. We call the access to credentials over some time the \textit{Proof of Time}. We build simulators and analyzed the proof-of-time implementation based on graph infections proposed in \cite{jekon2019content} and found some interesting observations. The graph infections algorithm struggle with the Defensive Alliance problem described in Section \ref{observations}. Also, we claim doubt if its configuration (parameters $\xi$ and $Z_{ia}$) sensitivity is suitable for internet level networks where the structure can change. We also point out the lack of both fail-stop and Byzantine-fault tolerance property. Then we generalize the problem to a distributed system consensus problem and propose a solution based on blockchain technology. Besides the fact that blockchain as a distributed system solves the consensus problem, it also offers valuable properties in terms of trust and security such as immutability, neutrality, and security. We propose a scheme in which the content authentication mechanism is embedded not in the consensus mechanism, but as an overlay structure on top of the consensus algorithm. In consequence, we gain the deterministic and progressive authentication. We found Stellar and its Federated Byzantine Agreement consensus protocol to be the most suitable blockchain implementation to our needs, it doesn't require mining or any other market forces to achieve security. One disadvantage is the disjoint-quorums problem described in Section \ref{FBA}. 

We believe that authentication schemas based on proof-of-time may become the valuable option for future development of ICN networks. 

