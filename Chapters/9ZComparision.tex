\chapter{Comparison}
\label{comparison}
In comparing the two presented approaches, GI and Blockchain, we have focused on five main characteristics referred to as outcome, resilience, communication complexity, processing scalability (measured by the pieces of content the entire system can handle at a time), and fault-tolerance. One also has to consider the implementation requirements in terms of overlay IT infrastructure. We present the comparison in Table \ref{table:comparison}.
\begin{itemize}

    \item Outcome -- GI is indeterministic when it comes to a consensus decision. Even a high number of proof-of-times can sometimes result in an extinction, which is an unwanted property. Blockchain, on the other hand, is deterministic; a publisher with a high number of proof-of-time claims can be sure that his/her content will be authenticated. This is guaranteed by the safety property of FBA.
    
    \item Resilience -- GI is sensitive to both chance, the network size and structure, and the $\xi$, and $Z_{IA}$ parameters. While Blockchain––particularly using FBA––has just one assumption regarding the quorums-intersection property. We state that Blockchain is more resilient.
    
    \item Communication complexity -- In GI, each node has to communicate with just a constant number of its neighbors, so the complexity is linear in the number of nodes––$O(|N|)$. In Blockchain, each node has to communicate with all other nodes in the network, so the communication complexity is quadratic in the number of nodes––$O(|N|^2)$. In this regard, GI prevails.
    
    \item Processing capability -- to achieve safety, Blockchain has each network node process each piece of content sequentially, so the system is as fast as a single node in the network. Hence, adding nodes does not increase the total processing capability, which therefore is constant in the number of network nodes. In GI, communicates only with its neighbors. Therefore, nodes that are not connected can process different pieces of content independently--more network nodes increase the total processing capability, which is therefore linear in the number of network nodes.
    
    \item Fault-tolerance -- GI in its raw form does not handle fail-stop, let alone Byzantine faults. Blockchain tolerates up to $\frac{|N|-1}{3}$ Byzantine nodes, similarly as Stellar FBA. Blockchain therefore outperforms GI in this regard.
    
\end{itemize}

\begin{table}[h!]
\centering
\begin{tabular}{ || m{3cm} | m{5cm}| m{5cm} || }
\hline
\hline
Property & GI & Blockchain \\
\hline\hline
Outcome & Indeterministic & Deterministic \\
\hline
Resilience & Depends on chance, $|N|$, $\xi$ and $Z_{IA}$ parameters, network structure (defensive alliance-sensitive) & Depends on network structure (quorums-intersection required) \\
\hline
Communication complexity & $O(|N|)$ & $O(|N|^2)$ \\
\hline
Processing capability & $O(|N|)$ & $O(1)$ \\
\hline
Fault-tolerance & None & Byzantine-fault tolerance up to $\frac{|N|-1}{3}$ faulty nodes \\
\hline
Required overlay infrastructure & NDN, PKI & NDN, PKI, and blockchain network \\
\hline
\end{tabular}
\caption{Comparison of GI and Blockchain}
\label{table:comparison}
\end{table}

\chapter{Summary}
We have elaborated on the Stolen-Credentials Cpntent Poisoning Attack problem proposed in \cite{konorski2019mitigating}. Assuming time- constrained operation of a potential malicious publisher, we have quantified the effectiveness of the existing graph infection-based GI protocol and proposed an alternative, blockchain-based approach. The presence of the time constraint allows to extend the authentication abstraction from \textit{access to credentials} to both \textit{access to credentials} and \textit{access to time}, coining the term \textit{Proof of Time}. We have developed simulators and analyzed the proof-of-time implementation based on GI. Among others, we have observed that GI struggles with the Defensive Alliance problem described in Section \ref{observations}, and remarked on GI's sensitivity to the parameter configuration ($\xi$ and $Z_{ia}$), which may be troublesome in Internet-level graph structures can change in time. Then we have generalize the considerations to a distributed consensus problem and proposed a solution based on Blockchain technology. Besides solving the consensus problem, it also offers valuable properties such as security, integrity, immutability, and Byzantine-fault tolerance. We have proposed a scheme in which the content authentication mechanism is embedded not in the consensus mechanism, but as an overlay structure (blockchain) on top of the consensus algorithm. In consequence, we gain a deterministic and reliable authentication mechanism called Credibility Score. We have found Stellar and its Federated Byzantine Agreement consensus protocol to be a blockchain implementation most suitable for our needs. One disadvantage of it is the quorum-intersection requirement described in Section \ref{FBA}. Additionally, the Blockchain solution is more demanding in terms of overlay infrastructure and does not scale well with the network size.
We believe that authentication schemes based on proof-of-time may become a valuable option for the future development of ICN environments.