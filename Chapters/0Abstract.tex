\chapter*{Abstract}
Information-centric networks––that are candidates for the title of the network of the future––struggle with a new vector of attacks. One of them is content poisoning, especially the stronger form––Fake Data CPA. ICN networks are especially prone to those types of attacks due to a lack of removal functionality. Currently, known authentication methods like login/password, private key, biometry, SMS/email confirmation operate on the same dimension of authentication. We propose a new authentication dimension, which is time availability. When an adversary publisher is operating in a time-constrained environment, his access to target credentials is limited, whereas honest publisher is not timely constrained in any way. We leverage such distinction to propose a new authentication mechanism.
Two implementations are proposed, the first one based on infection processes in graphs, and the second one based on blockchain technology. We provide simulators that give us interesting observations, such as the Defensive Alliance problem in graph infection algorithm. We evaluate both approaches in terms of communication complexity, scalability, determinism, resilience, and fault-tolerance (including its stronger form––Byzantine-fault).