\chapter*{Streszczenie 2}
Praca została rozpoczęta od wprowadzenia do sieci Informacjo-centrycznych. Zasada działania została opisana na przykładzie projektu Named Data Networking (NDN). Następnie przedstawione zostały problemy z jakimi zmagają się sieci ICN. Jednym z nich jest atak zatruwania treści (ang. Content Posioning Attack (CPA)), a zwłaszcza jego silniejsza forma––atak zatruwania treści przez fałszywe dane (ang. Fake Data CPA). W celu inspiracji wykonany został przegląd innych technologii które również zmagają się z problemem zatruwania treści, a dokładniej Wikipedia, LOCKSS, oraz media społecznościowe. Te ostatnie pozwalają dostrzec podobieństwa między Fake New-sami, a zatruwaniem treści. Zaproponowane zostało rozwiązane problemu Fake News-ów bazujące na czasie dostępu Fake News-a w miejscu w którym jego żywotność jest ograniczona. Następnie rozwiązanie to zostało uogólnione do metody uwierzytelniania treści, nie tylko w mediach społecznościowych, ale również w sieciach ICN. Uogólnione rozwiązanie nazwane Dowód-Czasu (ang. Proof-of-Time) polega na rozszerzeniu mechanizmu uwierzytelniania treści o składnik dostępu czasowego do uwierzytelnień pozwalających na publikacje danych.
Następnie bazując na pracy \cite{konorski2019mitigating} zbadany został algorytm infekcji na grafach w celu implementacji wcześniej założonego mechanizmu. Przebadane zostały różne modele używane w epidemiologii które pozwalają na symulację oraz analizę procesów epidemiologicznych.
Kolejno przebadane zostały struktury grafów oraz algorytmy genratorów pozwalające na budowę sieci o różnych właściwościach m.in. sieci bezskalowe. 

Kolejny rozdział poświęcony został przeglądowi ataków na dane uwierzytelniające. Informacja ta, pozwala na oszacowanie wymaganego dostępu czasowego do uwierzytelnień w celu efektywnego działania mechanizmu Dowodu-Czasu.

Następny rozdział szczegółowo opisuje zasade działania algorytmu infekcji na grafach, w szczególności opisuje maszyne stanów która steruje zachowaniem każdego węzła w sieci. Opisuje również mechanizm rozprzestrzeniania się infekcji.

Stworzone na cele tej pracy simulatory pozwalają na dogłębną analizę algorytmu, oraz na obserwacje anomalii. Jedną z takich anomali jest zjawisko zwane Sojuszem Defensywnym, który skutecznie może zablokować rozprzestrzenianie się epidemii––co w przypadku tego algorytmu jest niechcianym efektem. 

Następny problem infekcji na grafach zostaje uogólniony do problemu konsensusu. Zauważone zostaje ze końcowy stan sieci -- epidemia lub wygaśnięcie, może być interpretowany jako decyzja wypracowana na drodze konsensusu wszystkich węzłów sieci. Z tegorozdział generalizuje problem infekcji na grafach do problemu konsensusu. Wezły powodu problem ten może być rozwiązany używając również innych algorytmów konsensusu.

Zaproponowany został 

We started this thesis from formulating the problem of Fake Data CPA. It turns out that Fake Data CPA is a hard problem, but assuming certain conditions (time constraints), we proved that it is solvable. Time constrain allows us to extend the authentication from \textit{access to credentials} to both \textit{access to credentials} and \textit{access to time}. We call the access to credentials over some time the \textit{Proof of Time}. We build simulators and analyzed the proof-of-time implementation based on graph infections proposed in \cite{konorski2019mitigating} and found some interesting observations. The graph infections algorithm struggle with the Defensive Alliance problem described in Section \ref{observations}. Also, we doubt if its configuration (parameters $\xi$ and $Z_{ia}$) sensitivity is suitable for internet level networks where the structure can change. We also point out the lack of both fail-stop and Byzantine-fault tolerance property. Then we generalize the problem to a distributed system consensus problem and propose a solution based on blockchain technology. Besides the fact that blockchain as a distributed system solves the consensus problem, it also offers valuable properties such as immutability and security. We propose a scheme in which the content authentication mechanism is embedded not in the consensus mechanism, but as an overlay structure on top of the consensus algorithm. In consequence, we gain the deterministic and progressive authentication. We found Stellar and its Federated Byzantine Agreement consensus protocol to be the most suitable blockchain implementation to our needs, it does not require mining or any other market forces to achieve security. One disadvantage is the disjoint-quorums problem described in Section \ref{FBA}. Additionally, the blockchain solution is more complex, introduces more communication overhead, and does not scale horizontally.

We believe that authentication schemas based on proof-of-time may become a valuable option for the future development of ICN networks. 

