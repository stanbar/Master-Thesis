\chapter*{Streszczenie 2}
Praca dotyczy bezpieczeństwa informacji w sieciach informacjocentrycznych (ang. Information-Centric Networks, ICN) na przykładzie technologii dystrybucji treści pod nazwą Named Data Networking (NDN). Jednym z problemów jest atak zatruwania treści (ang. Content Posioning Attack, CPA), a zwłaszcza jego silniejsza forma––atak zatruwania treści przez wydawców posugujcych si kradzionymi danymi uwierzytelniajcymi (ang. Stolen-Credentials CPA). Dla czytelniejszego wprowadzenia wykonany został przegląd innych technologii zmagających się z tym problemem, w tym Wikipedii, LOCKSS, oraz technologii mediów społecznościowych. Te ostatnie pozwalają dostrzec podobieństwa między znanym problemem fałszywych wiadomości (ang. fake news) a zatruwaniem treści. Zaproponowane zostało rozwiązane oparte na ograniczonym czasie dostępu do fałszywych wiadomości. Następnie rozwiązanie to zostało uogólnione na metody uwierzytelniania treści, nie tylko w mediach społecznościowych, ale również w sieciach ICN. Uogólnione rozwiązanie, nazwane Dowód Czasu (ang. Proof-of-Time), polega na rozszerzeniu mechanizmu uwierzytelniania treści o składnik dostępu czasowego do uwierzytelnień pozwalających na publikacje danych.
Następnie bazując na pracy \cite{konorski2019mitigating} zbadany został algorytm infekcji na grafach w celu implementacji wcześniej założonego mechanizmu. Przebadane zostały różne modele używane w epidemiologii, które pozwalają na symulację oraz analizę procesów epidemiologicznych.
Kolejno przebadane zostały struktury grafów oraz algorytmy generatorów pozwalające na budowę sieci o różnych właściwościach, m. in. o strukturze grafów przypadkowych i bezskalowych.

Kolejny rozdział poświęcony został przeglądowi ataków na dane uwierzytelniające. Informacja ta pozwala na oszacowanie wymaganego dostępu czasowego do uwierzytelnień w celu efektywnego działania mechanizmu Dowodu Czasu.

Następny rozdział szczegółowo opisuje zasadę działania algorytmu infekcji na grafach, w szczególności opisuje maszynę stanów, która steruje zachowaniem każdego węzła w sieci. Opisuje również mechanizm rozprzestrzeniania się infekcji.

Stworzone dla celów tej pracy symulatory pozwalają na dogłębną ilościową analizę algorytmu oraz na obserwacje rozmaitych anomalii. Jedną z nich jest zjawisko nazwane Sojuszem Defensywnym, który może skutecznie zablokować rozprzestrzenianie się epidemii––co w przypadku tego algorytmu jest efektem niepożądanym .

Następnie problem infekcji na grafach został uogólniony do problemu konsensusu. Zauważone został, że końcowy stan sieci -- epidemia lub wygaśnięcie infekcji, może być interpretowany jako decyzja wypracowana na drodze konsensusu wszystkich węzłów sieci. Z tego powodu problem ten może być rozwiązany z użycien znanych algorytmów konsensusu w systemach rozproszonych.

Zaproponowane zostało rozwiązanie oparte o technologię blockchain, które poza rozwiązaniem problemu konsensusu, dostarcza wartościowych cech, takich jak odporność na błędy bizantyjskie (ang. Byzantine-fault tolerance) oraz uniezależnienie od topologi sieci. Dodatkowo rozwiązanie to pozwala na wprowadzenie progresywnego uwierzytelniania treści––w przeciwieństwie do algorytmu infekcji na grafach, gdzie dana treść mogła mieć jedynie status binarny--uwierzytelniona albo nieuwierzytelniona––w rozwiązaniu opartym na technologii blockchain możliwe jest elastyczne interpretowanie wiarygodności każdej treści z osobna. Treści typu muzyka, filmy oraz zdjęcia są mniej wrażliwe na ataki zatruwania treści, niż treści takie jak oprogramowanie, lub strony internetowe wymagające podania poufnych danych. System oparty technologii blockchain rejestruje długość Dowodu Czasu, pozostawiając jego interpretację końcowemu użytkownikowi.

Do dalszych badań wybrany został protokół konsensusu Federated Byzantine Agreement (FBA) ze względu na możliwość zabezpieczenia otwartych sieci zaufania (których węzły mogą w każdej chwili dołączać lub odłączać się), szybkość działania oraz tolerancję obecności do jednej trzeciej węzłów wykazujących zachowania bizantyjskie. FBA wymaga, aby każde sformowane kworum (podzbiór węzłów) miało część wspólną z każdym innym kworum; niespełnienie tego warunku prowadzi do utraty spójności stanu sieci. Dodatkowo rozwiązanie oparte na technologii blockchain cechuje się większą komplikacją i gorszą skalowalnością wraz z liczbą węzłów sieci, a także wprowadza dodatkowe wymagania co do infrastruktury nakładkowej.

Na koniec pracy przedstawione zostaje porównanie obu podejść. Algorytm infekcji na grafach jest szybszy i prostszy, jednak algorytm oparty na technologii blockchain cechuje większy determinizm, odporność na błędy oraz elastyczność.