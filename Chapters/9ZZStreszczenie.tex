\chapter*{Streszczenie}
Sieci informacjo-centryczne (ang. Information-centric Networking (ICN))––które kandydują do miana sieci przyszłości––zmagają się z nowym wektorem ataków. Jednym z ataków jest atak zatruwania treści (ang. Content Posioning Attack (CPA)), a zwłaszcza jego silniejsza forma––atak zatruwania treści przez fałszywe dane (ang. Fake Data CPA). Sieci te, są wyjątkowo podatne na ten typ ataków, ponieważ nie pozwalają na usuwanie treści––w momencie, gdy treść trafia do sieci, nie jest możliwe jej usunięcie inaczej niż poprzez wygaśnięcie.
Aktualnie znane metody uwierzytelniania jak login/hasło, klucze prywatne, biometria, potwierdzenie SMS/email operują w jednym tym samym wymiarze uwierzytelniania. W tej pracy proponujemy nowy wymiar uwierzytelniania, który bazuje na dostępnie do czasu. Kiedy napastnik-wydawca operuje w czasowo ograniczonym środowisku, jego dostęp do skradzionych uwierzytelnień jest czasowo ograniczony, podczas gdy prawdziwy wydawca nie jest ograniczony czasowo w żaden sposób. Ta różnica jest wykorzystana w celu stworzenia nowego systemu uwierzytelniania.
Dwie implementacje systemu są zaproponowane, pierwsza bazuje na algorytmie infekcji na grafach, a druga na łańcuchu bloków (ang. Blockchain). Dzięki zbudowanym symulatorom jesteśmy w stanie zaobserwować problem Sojuszu obronnego (ang. Deffensive Alliance).
Obie metody zostają ocenione pod kątem złożoności komunikacyjnej, determinizmowi algorytmów, możliwości dopasowania się do zmian, oraz odporności na błędy (również silniejszej formy bizantyjskich generałów).





Praca została rozpoczęta od wprowadzenia czym są do sieci Informacjo-centrycznyche. Opisana została Zzasada działania została opisana na sieci ICN na przykładzie projektu Named Data Networking (NDN). Następnie przedstawione zostały problemy z jakimi zmagają się sieci ICN. a została oraz z jakimi problemami się mierzą.
Jednym z cięższych problemów jest atak zatruwania treści (ang. Content Posioning Attack (CPA)), a zwłaszcza jego silniejsza forma––atak zatruwania treści przez fałszywe dane (ang. Fake Data CPA). W celu inspiracji wykonany został przegląd innych technologi które również zmagają się z problemem zatruwania treści, a dokładniej Wikipedia, LOCKSS, oraz media społecznościowe. Te ostatnie okazują się 
S
Przyjrzeliśmy 
Następnie został sformułowany sposób w jaki chcemy rozwiązać ten problem

We started this thesis from formulating the problem of Fake Data CPA. It turns out that Fake Data CPA is a hard problem, but assuming certain conditions (time constraints), we proved that it is solvable. Time constrain allows us to extend the authentication from \textit{access to credentials} to both \textit{access to credentials} and \textit{access to time}. We call the access to credentials over some time the \textit{Proof of Time}. We build simulators and analyzed the proof-of-time implementation based on graph infections proposed in \cite{jekon2019content} and found some interesting observations. The graph infections algorithm struggle with the Defensive Alliance problem described in Section \ref{observations}. Also, we doubt if its configuration (parameters $\xi$ and $Z_{ia}$) sensitivity is suitable for internet level networks where the structure can change. We also point out the lack of both fail-stop and Byzantine-fault tolerance property. Then we generalize the problem to a distributed system consensus problem and propose a solution based on blockchain technology. Besides the fact that blockchain as a distributed system solves the consensus problem, it also offers valuable properties such as immutability and security. We propose a scheme in which the content authentication mechanism is embedded not in the consensus mechanism, but as an overlay structure on top of the consensus algorithm. In consequence, we gain the deterministic and progressive authentication. We found Stellar and its Federated Byzantine Agreement consensus protocol to be the most suitable blockchain implementation to our needs, it does not require mining or any other market forces to achieve security. One disadvantage is the disjoint-quorums problem described in Section \ref{FBA}. Additionally, the blockchain solution is more complex, introduces more communication overhead, and does not scale horizontally.

We believe that authentication schemas based on proof-of-time may become a valuable option for the future development of ICN networks. 

